\documentclass[11pt, a4paper]{csiroreport2012}

\usepackage{graphicx}
\usepackage{amssymb}
\usepackage{amsmath}
\usepackage{booktabs}
\usepackage[small]{caption}
\usepackage[authoryear]{natbib}
\usepackage{authblk}


\setlength{\belowcaptionskip}{\abovecaptionskip} % Space below table caption same as that above
									      % figure caption
\docdivision[ENERGY FLAGSHIP]
\doctitle[{\huge NUMBAT} \\ High-resolution simulations of density-driven convective mixing\\User manual]
\docauthors[\vspace{1cm} Christopher P.  Green and Jonathan Ennis-King]
\docreportnum[Report Number ****]
\docreportdate[\today]
\doccopyrightyear[2015]

\docfootertitle[Numbat user manual]

\docbusinessunit[CSIRO Energy Flagship \\
71 Normanby Road, Clayton VIC, 3168, Australia \\
Private Bag 10, Clayton South VIC, 3169, Australia \\
Telephone: +61 3 9545 2777 \\
Fax: +61 3 9545 8380]

\docfurtherinfoA[CSIRO Energy Flagship]{Chris Green}
{+61 3 9545 8371}{chris.green@csiro.au}{}{www.csiro.au}

\begin{document}

\section{Introduction}

Numbat is a finite element application for solving the coupled Darcy and convection-diffusion equations for density-driven convective mixing in porous media. Numbat is built on the MOOSE Framework (www.mooseframework.com), and leverages multiple powerful features from this foundation. It features mesh adaptivity, high-order finite elements, is massively parallel, and uses a simple plain text input file. 

\section{Theory}

The governing equations for density-driven flow in porous media are Darcy's law
\begin{equation}
\mathbf{u} = - \frac{\mathbf{K}}{\mu} \left(\nabla P - \rho(c) g \hat{\mathbf{z}} \right),
\label{eq:darcy}
\end{equation}
where $\mathbf{u}$ is the velocity vector, $P$ is the fluid pressure, $\rho(c)$ is the fluid density as a function of solute concentration $c$, $g$ is gravity, and $\hat{\mathbf{z}}$ is the unit vector in the $z$ direction.

The fluid velocity must also satisfy the continuity equation
\begin{equation}
\nabla \cdot \mathbf{u} = 0,
\end{equation}
and the solute concentration is governed by the convection - diffusion equation
\begin{equation}
\phi \frac{\partial c}{\partial t} + \mathbf{u} \cdot \nabla c = \phi D \nabla^2 c,
\label{eq:convdiff}
\end{equation}
where $\phi$ is the porosity, $t$ is time and $D$ is the diffusivity. 

Darcy's law and the convection-diffusion equations are coupled through the fluid density, which is given by
\begin{equation}
\rho(c) = \rho_0 + \frac{c}{c_0} \Delta \rho,
\label{eq:density}
\end{equation}
where $c_0$ is the equilibrium concentration, and $\Delta \rho$ is the increase in density of the fluid at equilibrium concentration.

\section{Installation}



\subsubsection*{Install MOOSE}
 Numbat is  a MOOSE application. In order to install Numbat, the MOOSE framework must first be installed. Detailed instructions are available at www.mooseframework.com/getting-started.
 
 \subsubsection*{Clone Numbat}
 
 The next step is to clone Numbat from GitHub. In the following, it assumed that MOOSE was installed to the directory \texttt{$\sim$/projects}. If MOOSE was installed to a different directory, the following instructions must be modified accordingly.

To clone Numbat, use the following commands at the command line:

\begin{verbatim}
cd ~/projects
git clone https://github.com/cpgr/numbat.git
cd numbat
git checkout master
\end{verbatim}

At this stage, there should be a \texttt{$\sim$/projects/numbat} directory.

\subsubsection*{Compile Numbat}
Next, compile Numbat using
\begin{verbatim}
make -jn
\end{verbatim}
where \texttt{n} is the number of processing cores on the computer. If everything has gone well, Numbat should compile without error, resulting in a binary named \texttt{numbat-opt}.

\subsubsection*{Test Numbat}
Finally, to test that the installation worked, the small test suite can be run using
\begin{verbatim}
./run_tests -jn
\end{verbatim}
where \texttt{n} is the number of processing cores on the computer. If everything has worked, the automatic tests should run and pass, and you are ready to use Numbat to undertake high-resolution simulations of density-driven convective mixing in porous media.

\section{Input file syntax}

The input file for a Numbat simulation is a simple hierarchical, block-structured plain text file identical to the MOOSE input file. 

\subsection{2D models}

The main blocks required to implement a 2D simulation of density-driven convective mixing are now discussed.
\subsubsection*{Variables}

For a 2D model, the simulation must have two variables: a \emph{concentration} variable, and a \emph{streamfunction} variable. These can be implemented in the input file using the following code:

\begin{verbatim}
[Variables]  
  [./concentration]  
  [../]  
  [./streamfunction]  
  [../]  
[]
\end{verbatim}

\subsubsection*{Kernels}

The kernels block are where the physics of the problem are specified. Three individual kernels are required for a 2D model: a \emph{DarcyDDC} kernel for the \emph{streamfunction} variable, a \emph{ConvectionDiffusionDDC} kernel for the \emph{concentration} variable, and a \emph{TimeDerivative} kernel also for the \emph{concentration} variable. An example for an isotropic model is

\begin{verbatim}
[Kernels]
  [./TwoDDarcyDDC]
    type = DarcyDDC
    variable = streamfunction
    concentration_variable = concentration
  [../]
  [./TwoDConvectionDiffusionDDC]
    type = ConvectionDiffusionDDC
    variable = concentration
    streamfunction_variable = streamfunction
    coeff_tensor = '1 0 0 0 1 0 0 0 1'
  [../]
  [./TimeDerivative]
    type = TimeDerivative
    variable = concentration
  [../]
[]
\end{verbatim}

\subsubsection*{AuxVariables}

The velocity components in the $x$ and $y$ directions in a 2D model can be calculated using the auxiliary system. These velocity components are calculated using the \emph{streamfunction}, see the mathematical model for details.

In the 2D case, two auxiliary variables, $u$ and $w$, can be defined for the horizontal and vertical velocity components, respectively. Importantly, these auxiliary variables must have \emph{constant monomial} shape functions (these are referred to as elemental variables, as the value is constant over each mesh element). This restriction is due to the gradient of the \emph{streamfunction} variable being undefined for nodal auxiliary variables (for example, those using linear Lagrange shape functions). Auxilliary variables for the velocity components can be defined using
\begin{verbatim}
[AuxVariables]
  [./u]
    order = CONSTANT
    family = MONOMIAL
  [../]
  [./w]
    order = CONSTANT
    family = MONOMIAL
  [../]
[]
\end{verbatim}

\subsubsection*{AuxKernels}

The velocity components are calculated by \emph{VelocityDDCAux} AuxKernels, one for each component. For the 2D case, the input syntax is
\begin{verbatim}
[AuxKernels]
  [./uAux]
    type = VelocityDDCAux
    variable = u
    component = x
    streamfunction_variable = streamfunction
  [../]
  [./wAux]
    type = VelocityDDCAux
    variable = w
    component = y
    streamfunction_variable = streamfunction
  [../]
[]
\end{verbatim}

\subsection{3D models}

\subsubsection*{Variables}

For a 3D model, three variables are required: one \emph{concentration} variable and two \emph{streamfunction} variables corresponding to the $x$ and $y$ components. This can be implemented in the input file using:
\begin{verbatim}
[Variables]  
  [./concentration]  
  [../]  
  [./streamfunctionx]  
  [../]  
  [./streamfunctiony]  
  [../]  
[]
\end{verbatim}

\subsubsection*{Kernels}

Four individual kernels are required for a 3D model: a \emph{DarcyDDC} kernel for each \emph{streamfunction} variables, a \emph{ConvectionDiffusionDDC} kernel for the \emph{concentration} variable, and a \emph{TimeDerivative} kernel also for the \emph{concentration} variable. An example of the kernels block for a 3D isotropic model is
\begin{verbatim}
[Kernels]
  [./ThreeDDarcyDDCx]
    type = DarcyDDC
    variable = streamfunctionx
    concentration_variable = concentration
    component = x
  [../]
  [./ThreeDDarcyDDCy]
    type = DarcyDDC
    variable = streamfunctiony
    concentration_variable = concentration
    component = y
  [../]
  [./ThreeDConvectionDiffusionDDC]
    type = ConvectionDiffusionDDC
    variable = concentration
    streamfunction_variable = 'streamfunctionx streamfunctiony'
    coeff_tensor = '1 0 0 0 1 0 0 0 1'
  [../]
  [./TimeDerivative]
    type = TimeDerivative
    variable = concentration
  [../]
[]
\end{verbatim}

In the 3D case, it is important to note that the \emph{DarcyDDC} kernel must specify the component that it applies to, and that the \emph{streamfunction\_variable} keyword in the \emph{ConvectionDiffusionDDC} kernel must contain both \emph{streamfunction} variables ordered by the $x$ component then the $y$ component.

\subsubsection*{AuxVariables}

For the 3D case, there is an additional horizontal velocity component ($v$), so the input syntax is
\begin{verbatim}
[AuxVariables]
  [./u]
    order = CONSTANT
    family = MONOMIAL
  [../]
  [./v]
    order = CONSTANT
    family = MONOMIAL
  [../]
  [./w]
    order = CONSTANT
    family = MONOMIAL
  [../]
[]
\end{verbatim}

\subsubsection*{AuxKernels}

For the 3D case, three \emph{AuxKernels} are required. Note that both \emph{streamfunction} variables must be given, in the correct order ($x$ then $y$). 
\begin{verbatim}
[AuxKernels]
  [./uAux]
    type = VelocityDDCAux
    variable = u
    component = x
    streamfunction_variable = 'streamfunctionx streamfunctiony'
  [../]
  [./vAux]
    type = VelocityDDCAux
    variable = v
    component = y
    streamfunction_variable = 'streamfunctionx streamfunctiony'
  [../]
  [./wAux]
    type = VelocityDDCAux
    variable = w
    component = z
    streamfunction_variable = 'streamfunctionx streamfunctiony'
  [../]
[]
\end{verbatim}



\end{document}